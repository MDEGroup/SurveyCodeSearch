\lstset{
	basicstyle=\small\ttfamily,
	columns=flexible,
	breaklines=true,
	alsoletter={+},
}

% Minimal (M3-Rascal) syntax.
\newcommand*{\irsc}[1]{\lstinline[language=m3]+#1+} % inline manifest syntax
\lstdefinelanguage{m3}{
	basicstyle=\ttfamily\scriptsize,
	keywordstyle=\bfseries,
	keywords={m3,declarations,methodInvocation},
	literate={<-}{$\leftarrow$}{1},
	tabsize=2,
	alsoletter={-}
}

\newcommand{\code}[1]{{\small \texttt{#1}}}

\newcommand*\circled[1]{\tikz[baseline=(char.base)]{\color{black} 
		\node[shape=circle,draw=cyan,fill=black!10!white,inner sep=.3pt] (char) {{{\texttt\textbf #1}}};}}
\def\checkmark{\tikz\fill[scale=0.4](0,.35) -- (.25,0) -- (1,.7) -- (.25,.15) -- cycle;} 


% Usual suspects
\usepackage{xspace}
\newcommand*{\ie}{i.e.,\@\xspace}
\newcommand*{\eg}{e.g.,\@\xspace}
\newcommand*{\cf}{cf.\@\xspace}
\makeatletter
\newcommand*{\etc}{%
	\@ifnextchar{.}%
	{etc}%
	{etc.\@\xspace}%
}
\makeatother
\newcommand*{\etal}{et~al.\@\xspace}

\newcommand{\nb}[2]{
	\fbox{\bfseries\sffamily\scriptsize#1}
	{\sf\small$\blacktriangleright$\textit{#2}$\blacktriangleleft$}
}
\newcommand\MAX[1]{\textcolor{blue}{\nb{MAX}{#1}}}
\newcommand\TD[1]{\textcolor{cyan}{\nb{TD}{#1}}}
\newcommand\LO[1]{\textcolor{green}{\nb{LO}{#1}}}
\newcommand\JDR[1]{\textcolor{orange}{\nb{JDR}{#1}}}
\newcommand\PN[1]{\textcolor{gray}{\nb{PN}{#1}}}