The problem of providing good enough recommendations is a common and recent issue that arises in the computer science community, carried by the introducing and growing of complex software systems. The common issues in this domain are the uncertainty of the user's needs, the correct statement of the context of the project that she is developing and the accuracy of the provided recommendation. The crucial point is to give a recommendation, in form of a snippet of code, that can help the developer, trying to reduce the following trade-off: provide too fewer recommendations means that the tool is not effective; on the other hand, give too many results is equal to not give any information at all. Many authors develop a recommendation system as a code search engine, by setting similarity measure and using information retrieval techniques. Focusing on this kind of recommendation tools, we want to contribute by proposing a novel approach that extends they and achieves better results in average.
We organized the work in this way:  \Cref{sec:Introduction} gives information about the domain in which we are and the problem of recommend source code. \Cref{sec:Challenges} exhibits the opportunities and the challenges in this domain and the possible way to address them.
\Cref{sec:Features} describes what are the archetypical features that we want to support.  \Cref{sec:RelatedWorks} offers an overview of the existing approaches and how they approach the main issues. \Cref{sec:ProposedApproach} shows our approach and the novelty that we bring.  Finally,  \Cref{sec:Conclusion} discusses the threats and possible future works in this area. 
