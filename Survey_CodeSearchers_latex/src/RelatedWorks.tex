At this point, we show an overview on the existing approaches by pointing out the technique and the main features in order to do the comparison with our approach. 

\subsection{Prompter}
We look at \cite{DBLP:journals/ese/PonzanelliBPOL16} in which the authors propose Prompter, an automatic tool to search and retrieve recommendations coming from SO. The key idea is to perform a web crawling activity in Google considering the developer's context. Given this context, composed by the source code, Prompter builds the query using the Query Generator Service (QGS) component. 
 To build the query, the authors use a naive approach, presented by Ponzanelli et al., that splits the identifiers and removes the stop word, ranks the terms according to their frequency and shows the top rank element. Moreover, they consider also the entropy of each term, according to Shannon's interpretation. Thus, Prompter uses the text quality index (TDI) to assess the goodness of the query, by considering the features described above. Once the proper component builds the query,  the QGS component sends it, with the context, to the Search Service that acts as an interface among the tool, the web engines (ie Google, Bing) and StackOverflow API that performs the crawling of the discussion. Finally, the authors use a ranking model to sort the results according to different metrics. We will see them in details as they represent a baseline to evaluate our approach. 
Concerning the final evaluation of the produced recommendations by Prompter, the authors made two studies to assess the quality of the tool: one relies on a survey in which the authors ask 55 people (with different skills and job experiences) to evaluate the Prompter's results in term of accuracy. In particular, the questions want to discover how much the developers use StackOverflow as support during the development (as well as Java API documentation or similar) and if the Prompter's suggestions are related (or not) to the context. 
The second study involves instead several programming tasks that the developer have to implement. Going in deep, the authors propose two kinds of task: a maintenance task (MT), that requires only little changes in the original code, and a development task (DT) in which the developer must implement multiple functionalities from scratch. The results claim that Prompter works well and, in particular, the developers indicate as valuable features the accuracy of suggestion and the user interface. On the other hand, the main limitations are the interface that doesn't allow to choose key terms during the search activity and the missing integration of a  search field, that forces the developer to exit from IDE and surf on Web. To validate all these results, the authors use R studio to apply statistical methods and functions. 


\subsection{FaCoY – A Code-to-Code Search Engine}
In this work \cite{DBLP:conf/icse/KimKBC0KT18}, the authors propose a tool, Facoy,  to search similarity in snippets of code. Facoy uses Lucene to build and perform a query (and alternate versions of it) on SO dump. The aim is to find similar posts on SO considering the semantic and not only the syntactic structure of the snippet of code.  To do this,  the authors use Lucene to set indexes that represent the relationship between the code and the semantic. In particular, Facoy uses the snippet index to associate the answer document IDs with the code in the SO posts, by creating pairs in form of token\_type: actual\_token with Lucene. To generate all possible terms from the code, the authors parse the AST of the code, by wrapping the incomplete code in a dummy class if needed. The second index, called question index, maps the information between the question and the source code in the SO posts. Facoy uses this index to perform semantic analysis and try to find alternate queries that should cover more cases. Finally, Facoy uses the code index to search the similarity between the user's code and the code in the SO posts.Notice that the SO posts are used only to learn and the recommendation baseline are the Github projects with most elements in common with the context. The tool builds the AST of the code (building dummy classes if needed), similarly with respect to the first index. Once the authors define these indexes, Facoy retrieves the results and, by using Lucene functions, it ranks the results. In particular, the tool exploits the Lucene function for textual similarities like Boolean Model and Vector Space model for scoring, plus TF-IDF for rank the results and to compute the Cosine similarity between the developer's code and indexed snippets. So, the final results are snippet of code coming from Github projects related to the context. To evaluate the obtained results and assess the quality of the proposed tool, the authors set four different research questions about the precision and accuracy of Facoy against the other code search engines. As dataset, they consider 10.449 Github projects and the SO dump from 2008 to 2016 to learn the relationship among the token. They focus on Java and Android code and index over 1.800.000 SO posts. From the evaluation, Facoy outperforms Krugle and Searchcode, the two code search engine considered in the comparison. 

\subsection{CodEX: Source Code Plagiarism Detection Based on Abstract Syntax Trees}
Codex \cite{DBLP:conf/aics/ZhengPL18}, the proposed tool, analyzes the AST of the code and perform a similarity analysis based on fingerprints. In facts, it summarizes the AST of the code and then generates the respective fingerprint to represent relevant information of the tree. In particular, Codex gives a specific weight to each fingerprint, although it performs a relatively fine-grained analysis that consists in giving a lower score to so-called tiny terms (i.e. parameters or constant) that have less impact than bigger nodes in the tree. Moreover, the authors analyze also the occurrence of sibling functions and, finally, compare the fingerprints of the subtrees in a top-down manner. Considering these weights, Codex performs the search considering two levels of similarity: the local one, that compare the pair context-single snippet, and the global one, in which it compares the fingerprint of the developer's code with the whole corpus. To make these comparisons, the tool walks the ASTs and prunes unnecessary nodes, according to a given threshold. To assess the quality of Codex, the authors select 10 test cases with plagiarised code, 5 written in Python and 5 in Java. In particular, they consider different kinds of plagiarism, like code modification, comments insertion, block combination. The results show that Codex is able to recognize the duplicated code in almost test cases, with accuracy near to 90\%.

\subsection{CodeHow: Effective Code Search based on API Understanding and Extended Boolean Model}
In this work \cite{DBLP:conf/kbse/LvZLWZZ15}, the authors propose a code search engine, called CodeHow, that perform the code analysis by using two key concepts: the API documentation and the Extended Boolean model. Focusing on the whole process, CodeHow parsers the API online documentation by analyzing the user query. As the first step, the tool retrieves and parser the information coming from the documentation using the classical text preprocessing techniques (text normalization, stop words removal and stemming). In this way, CodeHow is able to find similarities between the user’s query and the API related to it. To identify the relevant APIs among all retrieved ones, the authors propose a different information retrieval technique, called the Extended Boolean Model (EBM). From the literature, we know that there are two main techniques to asses the similarity of two elements: the Boolean model, that assigns 0,1 as weights to elements, and the Vector Space Model (VSM), that summarizes documents using vectors and calculate usually the Cosine similarity on them. The EBM is a trade-off between the two previous techniques because it assigns weights that belongs to an interval (between 0 and 1) and then normalizes the metric using the p-value (in this paper, the authors set it to 3).
Coming back to the proposed approach, they use Elastic Seach to index the results, the classical VSM function already built-in in the Apache Lucene library and, finally, the Extend Boolean Model to build the query that achieves more results. In particular, they analyze a single term before assigning it a weight: if the term t is an API, CodeHow assigns to it the API score calculated by Lucene VSM; if the term is not an API, the tool uses the TF-IDF index to set the score. In order to evaluate the obtained results, the authors use a dataset composed of 26.000 C\# projects. Considering this dataset, they perform first an internal experiment, by performing 34 real queries coming from the literature and checking by hand the top 20 results for each query. Then, they perform a user evaluation involving 20 Microsoft developers who are called to implement three different tasks: the first one regards the sending emails, the second concerns the conversion from text to XML and the last the image format conversion. The authors compare their approach with Ohloh, another code search engine, and, from this comparison, CodeHow performs the analysis better than the opponent. Moreover, they evaluate the results with MRR and precision metrics already seen in other works.  


\subsection{Sourcerer: mining and searching internet-scale software repositories}
Sourcerer \cite{DBLP:journals/datamine/LinsteadBNRLB09}, a tool that performs code search in the large-scale repositories. The first component involved in the processing is the crawler, that downloads automatically the repositories. Then, the tool analyzes the code involving three entities: the parser, that maps each repository to the entity and possible relationships, modeled by a relational database; Lucene, that indexes the entities using keywords included in the code. Beside this keyword-based technique, Sourcerer uses also the fingerprints, that summarize the snippets in vectors and give information about the syntactical information of the code. The tool uses them as support for structural search.  Finally, the third component is the ranker, based on the graph-based representation of the code to filter and retrieve top rank results, going beyond the text-based similarity. Additionally, it uses also fingerprints and keywords analysis, taking care of the best common practices in the computer science community (i.e name convention, analyzing the test classes, evaluating the complexity by LOC). Once Sourcerer made this kind of analysis and classification on the source code,  it maps also the authors of each snippet using a matrix with author-document entries. This kind of process gives further information about the developers who write the code: in particular, Sourcerer can categorize the authors with best skills according to their contributions. To evaluate the overall approach, the authors set up a user study on a dataset composed of over 4000 Java projects. They select 25 queries and do a manual evaluation  on the results, involving also three expert developers, considering four key aspects: the original content of the query and the topic of the search, the quality of the results in term of coverage, the reputation of the project and, finally, the possible reuse as a software component. Besides the human evaluation, they also use classical recall and precision metrics to assess the quality of the results from the statistical point of view. The results show that Sourcerer achieves high precision and accuracy considering the queries previously selected. 

\subsection{Retrieving Software Objects in an Example-Based Programming Environment}
In this work \cite{DBLP:conf/sigir/Henningen91}, the author proposes a tool, called CodeFinder, that performs recommendations using the example-based programming technique. As claimed in the paper, the most difficult part is to build the query and understand what the developer really needs. A system that relies only on a keyword-based system doesn't provide good enough results, because it excludes results that should be interesting for the user's purpose. Another problem strongly related to the example-based technique is that the suggested code doesn't fit exactly on the user's needs. To solve these issues, CodeFinder implements a hybrid approach that mixes the keyword-based style with information retrieval techniques, like the retrieval by reformulation that considers a subset of a concept instead try to match the exact one. To do this, the tool builds a network, that represents the conceptual model of the user, in which the keywords are induced by the query that the user is writing, using also the thesaurus to find synonyms that are more closer with respect to the query.  In the network, the keywords are the nodes and the relationship between them represent the level of likelihood. Once the user performs the first time the query, the CodeFinder interface retrieves other related keywords, the matching items and some code example of them. At this point, the interface allows adding new keywords in order to improve the query, exploiting the retrieval by reformulation technique. The key idea of this work is to guess the overall intentions of the developer when she writes some code, instead to reasoning only on the code or the query. Unfortunately, the validation of the tool is not available, as claimed by the author in the paper.

\subsection{Strathcona Example Recommendation Tool}
Finally, Strathcona \cite{DBLP:conf/sigsoft/HolmesWM05}, is a recommendation tool based on API that analyzes the developer's context from the structural point of view and suggests a possible implementation related to the task that she is developing. More in details, Strathcona uses six heuristics based on inheritance hierarchy, field types method calls and object usage in order to build the query. As the main similarity method, they use the structural context, based on the type of the classes and  type of their parents and fields, the signature of the methods while it doesn't consider the name of fields, parameters and 
This query is performed on a repository, called the example repository, that contains all possible usage of the APIs and it is built automatically from the context. Strathcona retrieves the code examples related to the query that the developer can navigate both graphically and in a textual way. At this point, she can select the code examples that are more related to its task or can query again after she has written more code, to have more specific examples. The overall approach is informally validated by the University of Calgary. The main issue is that a typical user doesn't know what is the starting point when she is implementing a completely new feature. The other issue is the quality of the example repository, that maybe not cover all possible cases. 







